\section{Introduction}
\label{sec:intro}

In recent years, the trend of applying Machine Learing (ML) concepts to improve decision making, predciting 
market sentiment, recognizing patterns, affective computing, or interpreting text have gained the attention of many. 
Within the realm of ML concepts are classification problems where the objective is to classify data entries 
as either categorical, ordinal, or binary \cite[p. 327-328]{textbook}. 
As a result, classification algorithms (CAs) were developed using statistical analysis techniques. 
When applied to a dataset, CAs can provide great insight into common features, known as class,
found within the data. Classifiers in particular use the training data to assess how the data's respective 
attributes fit within the definitions of a particular class with respect to the ground truth \cite{class}. 

A field of study known as Affective Computing attempts to infer the emotional state of a human being during computer usage. 
 Knowing the emotional state of the user can help machines alter thier 
content based on this information. Companies can use this on their employee's computers to help improve 
productivity and health. 

This year, a group of researchers introduced a WEarable Stress and Affect Detection (WESAD) dataset that 
provides a multimodal high-quality dataset with various affective states \cite{WESAD}. The experiment 
tested for three affective states amusement, stress, and neutral. It also could help determine whether a test 
subject was or was not stressed. In general, stress and emotion are not mutually exclusive since both 
rely on the body's central nervous system. This has been a problem within the Affective Computing 
research field. However, this dataset was made with the intention of bridging this problem. 
For the purpose of this paper, the focus is geared towards detecting if a user is stressed or not.  
