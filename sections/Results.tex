\section{Methods \& Results}
\label{sec:methods-results}

\subsection{Methods}
\label{sec:methods}

From the WESAD Dataset, the male patients were selected to train and test the SVM model. 
The dataset indexes each male or female patient as $S_n$ where $n \in [2,17]$ and 
$n \ne 12$ which yields a total of 15 patients. From $S_n$, the female patients were patients 
$n=8,11,12$. The resulting set of male only patients, $M$, is thus 
$$ M = \{S_n,n \ne 8,11,12,17\} $$
and has a cardinality of 12. The $M$ set was then remapped to fit a more intuitive 
indexing for SVM model and 12-Fold Cross Validation purposes. Table \ref{table:remapping}
denotes how each male patient's study index was remapped. 

\begin{table}[h!]
\centering
\begin{tabular}{||c c||} 
\hline
Study Index & Remapped Index  \\ [0.5ex] 
\hline \hline
$S_2$ & $M_1$  \\ 
\hline
$S_3$ & $M_2$  \\ 
\hline
$S_4$ & $M_3$  \\ 
\hline
$S_5$ & $M_4$  \\ 
\hline
$S_6$ & $M_5$  \\ 
\hline
$S_7$ & $M_6$  \\ 
\hline
$S_9$ & $M_7$  \\ 
\hline
$S_{10}$ & $M_8$  \\ 
\hline
$S_{13}$ & $M_9$  \\ 
\hline
$S_{14}$ & $M_{10}$  \\ 
\hline
$S_{15}$ & $M_{11}$  \\ 
\hline 
$S_{16}$ & $M_{12}$  \\ 
\hline 
\hline
\end{tabular}
\caption{Remapping WESAD indexes for implementation}
\label{table:remapping}
\end{table}

As noted in section \ref{sec:Problem-Statement}, a 12-Fold Cross Validation technique was chosen 
to partition the WESAD dataset into training and testing data. For the purpose of 
this experiment, each male is gaureenteed to be the training set and subsequently be 
the testing set. For example, $M_1$ would be the fitted as the first trainer in 
the SVM model and males $M_{2-12}$ would be tested using the fitted $M_1$ model. 
This process repeats itself a total of 12 times, once for each male. The average 
accuracies from each training and testing pair within each fold are recorded. 
To ensure that the training and teting data was of the same length each time, 
a function, \texttt{preprocessing}, was used to ensure both datum satisfied this condition.
At the end of each fold, meaning using the example above would be $M_1$ 
as trainer and $M_{12}$ as testing, the mean average is computed and recorded 
along with the remapped index. Finally the 12 final average and fold pairs 
are ranked in decsending order by average accuracy. The following pseduocode 
denotes how the methods above were expressed. 

\begin{algorithm}
\caption{12-Fold Cross Validation using SVM}\label{12FCV}
\begin{algorithmic}[1]
\scriptsize
\Require 12 male patients datum (EDA and HR)
\For{$M_i \in {M}$}
\State $\texttt{training} = [M_i^{EDA}, M_i^{HR}]$
\For{$M_j \in {M}$ and $i \ne j$} \Comment{Ignore case when training set is the testing set}
\State $\texttt{testing} = [M_j^{EDA}, M_j^{HR}]$
\If{$len(training) \ne len(testing)$}
\State $\texttt{training, testing} = preprocessing(training, testing)$
\EndIf
\State $\texttt{results} = [classify_{i,j}(training, testing)]$
\EndFor
\State $\texttt{averages} = (i,mean(results))$ \Comment{Append final averages after each fold}
\EndFor
\State $rank(averages, "descending")$
\end{algorithmic}
\end{algorithm}

In the interest of the reader, section \ref{sec:appendix} denotes the exact output of the 
results mentioned in this section.

\subsection{Results}
\label{sec:Results}

The table below shows the ranked results from the 12-Fold Cross Validation experiment.

\begin{table}[h!]
\centering
\begin{tabular}{||c c||} 
\hline
Male Patient & Accuracy (\%)  \\ [0.5ex] 
\hline \hline
$M_2$ & 74.673  \\ 
\hline
$M_1$ & 74.671  \\ 
\hline
$M_5$ & 74.626  \\ 
\hline
$M_7$ & 74.595  \\ 
\hline
$M_6$ & 74.593  \\ 
\hline
$M_8$ & 74.593  \\ 
\hline
$M_{4}$ & 74.592  \\ 
\hline
$M_{10}$ & 74.580  \\ 
\hline
$M_{9}$ & 74.574  \\ 
\hline
$M_{12}$ & 74.571\\ 
\hline 
$M_{11}$ & 74.564  \\ 
\hline
$M_{3}$ & 74.562  \\
\hline 
\hline
\end{tabular}
\caption{Final Averages after 12 Folds of testing}
\label{table:results}
\end{table}

The average computation time is 2 minutes and 19.480 seconds. This experiment was 
perfromed on a MacBook Pro, Late 2013, 6th Gen i7 Intel Core Processor, 16 GB 
RAM machine. Another experiment was perfromed on a HP x360 Spectre with an 8th Gen i7 Intel 
Core Processor, 16 GB RAM machine. The average computation time 
for this platform was is 2 minutes and 19.480 seconds. The 8th Generation 
Processor performed 31.13\% better than the 6th Generation one.
