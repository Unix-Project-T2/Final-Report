\section{Analysis}
\label{sec:analysis}

Before diving into the results noted in table \ref{tab:results}, here are some 
statistics about the males patients sampled from the WESAD dataset.

\begin{itemize}
	\item Average age: 27.5 yrs, Variance of 7.18 years
	\item Average height: 179.67 cm, Variance of 32.06 cm
	\item Average weight: 77 kg, Variance of 77 kg
	\item All were right hand dominant
	\item 16.67\% Drank coffee the day of their test
	\item No one drank coffee an hour before taking the exam
	\item 1 patient played sports the day of the exam
	\item 1 patient is a smoker
	\item No one smoked within an hour prior to the exam
	\item Only 1 patient felt ill the day of the exam
\end{itemize}
\bigskip 
$M_2$ proved to be the best trainer overall when testing against the other male patients 
with an average accuracy of NUM\%. Below is a list of attributes about $M_2$.

\begin{itemize}
	\item Right hand dominant
	\item Age: 27
	\item Weight: 66 kg
	\item Height: 129 cm
	\item Did not drink coffee the day of the study
	\item Did not drink coffee 1 hour before the study
	\item Did not participate in sport activities the day of the study
	\item Is not a smoker
	\item Did not smoke an hour before the study
	\item Did not feel ill the day of the study
	\item Additional Notes
	\begin{enumerate}
		\item During the baseline condition, the subject was sitting in a sunny workplace
		\item Subject provided a valence label of 7 after the stress condition, claiming that he was looking forward to the next condition and was therefore cheerful
	\end{enumerate}
\end{itemize}

The runtime for this experiment was exceedingly high. This is a result of single threading 
the entire experiment. A possible optimization would be to initiate 12 threads, one for 
each fold. The threads could run concurrently without the fear of stepping over one another. 
Since each fold simply steps through the dataset and selects their respective training and 
testing samples, a multithreaded solution would significantly increase runtime execution. 