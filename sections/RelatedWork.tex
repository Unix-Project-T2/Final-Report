\section{Related Work}
\label{sec:Related-Work}

The field of Affective Computing has picked up steam in the last couple of decades. 
This field of study is interdisciplinary which culminates Computer Science, Computer Engineering, Physcology, and 
Physiology. With the average employee spending 7 hours a day on the computer, 6 at work and 1 at home, 
it is important to monitor the health conditions of users during thier time behind a screen \cite{computer-time}.

Stress for the human body can affect health and wellness in a negative manner. Both men and women struggle with 
different facets of it. In particular, the measurement of these stress levels has been improved on beyond that of 
heart rates and blood pressure measurements. The Feminine Gender Role Stress (FGRS) scale, for example, 
was used to link women's eating disorders and stress. In 1995, a study concluded that 
women who have eating disorders also question their body image as a result of their gender role which yielded a 
higher FGRS rating. These women's FGRS levels were greater than the expected everyday stress for the average person 
\cite{eating-disorders}. A study using the Standard Stress Scale (SSS) rating has been used to determine the healthy 
and non-healthy amount of stress experienced in men. It concluded that the best determinant for 
high stress levels was through self-rated health \cite{stress-men}. 
It is important to also note that not all stress is bad. In fact, travel, love, graded exposure to fear and awkward 
situations, change, and being a beginner all yield positive stress for the human body \cite{good-stress}. 

Along with these stress level ratings comes error, bias, and variance in technology or simply, the subjects themselves.
The goal of modern Affective Computer lies within providing quality, accurate conclusions where other methods are not 
consistent in. Efforts in facial recognition that classify the affect of a human have been developing in recent 
years. The changes in muscles across the face lead to attributes that Charles Dawrin infered about universally about 
humans. These attributes have been consolidated to various class' that have been paired to classification algorithms 
for further analysis which resulted in 96\% accuracy for stationary images \cite{facial-recognition}. 

With the dependency of computers steadily increasing over time, the need for machines to interpret a user's affective state 
is important for their overall health. The contrary could result to ruined relationships or even shortness of life. 
This paper attempts to focus on stress detection during daily computer tasks to mitigate the negative effects of 
stress over time. 
